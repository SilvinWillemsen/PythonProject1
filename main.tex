\documentclass{article}
\usepackage[utf8]{inputenc}
\usepackage[left=3cm, right=3cm, top=2cm]{geometry}
\title{SCIENTIFIC COMPUTING USING PYTHON\\
1. PYTHON + SCIENTIFIC COMPUTING, PROJECT
}
\author{Silvin Willemsen}
\date{June 2021}

\usepackage{setspace}
\usepackage{graphicx}
\usepackage{appendix}
\usepackage{amsmath}
\usepackage{amsfonts}
\usepackage{amssymb}
\usepackage{listings}
\usepackage{algorithm2e}
\usepackage{xcolor}
\usepackage{dirtree}
\def\td{t_\delta}
\def\code#1{\texttt{#1}}

\begin{document}

\maketitle

\section{Problem Statement}\label{sec:problemStatement}
This projects implements the Lorenz attractor, devised by Edward Lorenz in his 1963 paper \cite{lorenz} in the Python programming language. The Lorenz attractor is a mathematical model for atmospheric convection and is described by the following non-linearly coupled ordinary differential equations (ODEs):
\begin{subequations}\label{eq:ODEsystem}
    \begin{align}
        \frac{dx}{dt} &= \sigma(y-x),\\
        \frac{dy}{dt} &= x(\rho - z) - y,\\
        \frac{dz}{dt} &= xy - \beta z,
    \end{align}
\end{subequations}
where independent variables $x = x(t)$, $y = y(t)$, and $z = z(t)$, can be interpreted as coordinates in three-dimensional space (in m) and are all dependent on time $t$ (in s). The system is parameterised by $\sigma$, $\rho$ and $\beta$, which govern the eventual behaviour of the system.

\subsubsection*{Discretisation}
In order to implement the continuous-time ODE described in \eqref{eq:ODEsystem}, they need to be approximated in a process called \textit{discretisation}. Time $t$ can be discretised using $t = n \td $. Here, the temporal index $n\in \{0, \hdots, N\}$ is an integer with desired number of iterations $N$ and $\td$ is the time step between two consecutive indices (in s). The spatial coordinates can then be be approximated using $x(t)\approxeq x[n]$, $y(t)\approxeq y[n]$ and $z(t)\approxeq z[n]$. 

To approximate the derivatives in \eqref{eq:ODEsystem}, we can use Euler's method. A derivative of a continuous function $f(t)$ can be approximated according to
\begin{equation}\label{eq:euler}
    \frac{df}{dt} \approxeq \frac{f(t+\td) - f(t)}{\td}.
\end{equation}
In other words, the derivative is approximated using difference between two consecutive function-values divided by the difference in $t$ ($\td$) between them. Other (more accurate) approximations exist, but for simplicity, we continue with the above.

We can then apply Eq. \eqref{eq:euler} to the system of ODEs in \eqref{eq:ODEsystem} to get an approximation. Together with substituting the discrete variables $x[n]$, $y[n]$ and $z[n]$, this yields
\begin{subequations}
\begin{align}
    \frac{x[n+1] - x[n]}{\td} &= \sigma(y[n]-x[n]),\\
        \frac{y[n+1] - y[n]}{\td} &= x[n](\rho - z[n]) - y[n],\\
        \frac{z[n+1] - z[n]}{\td} &= x[n]y[n] - \beta z[n].
\end{align}
\end{subequations}
The last step before these equations can be implemented in code, is to write them in terms of the respective coordinate at the next time step $n+1$. This yields the following update equations
\begin{subequations}\label{eq:updateEqs}
    \begin{align}
        x[n+1] &= \sigma(y[n]-x[n])\td + x[n],\\
        y[n+1]&= \Big( x[n](\rho - z[n]) - y[n]\Big) \td + y[n],\\
        z[n+1] &= (x[n]y[n] - \beta z[n])\td + z[n].
    \end{align}
\end{subequations}
which can be implemented in Python. A for-loop needs to be created so Eqs. \eqref{eq:updateEqs} can be iteratively calculated until a user-specified iteration number $N$.

\subsection{Cases}\label{sec:cases}
The project description states 5 different cases that need to be tested. The cases contain different values for the parameters $\sigma$, $\beta$ and $\rho$ according to

\begin{enumerate}
    \item $\sigma = 10$, $\beta = 8/3$, and $\rho = 6$
    \item $\sigma = 10$, $\beta = 8/3$, and $\rho = 16$
    \item $\sigma = 10$, $\beta = 8/3$, and $\rho = 28$
    \item $\sigma = 14$, $\beta = 8/3$, and $\rho = 28$
    \item $\sigma = 14$, $\beta = 13/3$, and $\rho = 28$
\end{enumerate}

\section{Implementation}
Algorithm \ref{alg:calcOrder} contains pseudocode showing the core algorithm implementing the Lorenz attractor. The implementation can be found online\footnote{https://github.com/SilvinWillemsen/lorenz}.

\begin{algorithm}[ht]
    \setstretch{1.1}
    \fbox{\parbox{0.8\linewidth}
    {

        Initialise the following:\\
        - $x[0]$, $y[0]$, and $z[0]$   \\  
        - free parameters $\sigma$, $\beta$ and $\rho$\\
        - time step $t_\delta$\\
        - no. of iterations $N$\\
        - vectors to store values for $x$, $y$ and $z$.\\
        \For {$n = 0:N-1$}
        {
            Calculate $x[n+1]$, $y[n+1]$, and $z[n+1]$ using Eqs. \eqref{eq:updateEqs}.\\
            Save these values to vectors initialised before
        }
        
        Plot the saved vectors $x$, $y$ and $z$.
        }
    }
    \vspace{0.12cm}
    \caption{\it Pseudocode showing the order of calculations.\label{alg:calcOrder}}
\end{algorithm}

The initial conditions used for the cases $x[0] = 1.0$, $y[0] = 1.0$ and $z[0] = 1.0$. Furthermore, the time step $\td = 0.002$ and number of iterations $N = 10000$ for all cases. A script is made where the simulation can be easily run using other initial conditions, and parameters. This will be introduced below.

\section{Design Considerations}
The goal of the project is to implement Eqs. \eqref{eq:updateEqs} in Python. The variables needed are the ones found in Eqs. \eqref{eq:updateEqs}. The output of the simulation needs to be three vectors (/arrays) that have the values of $x$, $y$ and $z$ stored over time. These can then be plotted using the \code{matplotlib.pyplot} module.

Below, the file and folder structure can be found:
\dirtree{%
.1 /.
.2 lorenz.
.3 \_\_init\_\_.py.
.3 run.py.  
.3 solver.py.  
.3 finiteDifferences.py.  
.3 plotting.py.
.2 cases.
.3 run\_case.py.
.3 generate\_data\_and\_figures.py.
.3 generate\_cases\_CSV.py.
.3 scheme\_variables.csv. 
.3 output\_files.
.4 case1\_output.
.5 simulation\_result.hdf5.
.5 xyPlot.pdf.
.5 xzPlot.pdf.
.5 yzPlot.pdf.
.5 xyzPlot.pdf.
.4 case2\_output.
.5 ....
.4 case3\_output.
.4 case4\_output.
.4 case5\_output.
.2 test.
.3 \_\_init\_\_.py.
.3 test\_lorenz.py.
.2 doc.
.3 ....
.2 LICENSE.
.2 README.txt.
}
\vspace{2em}
\noindent 
Below, some descriptions of the various files are given. A much more detailed description of the modules and scripts, however, can be found in the files themselves and via the sphinx-generated documentation in \code{doc/\_build/html/index.html}.

The \code{lorenz} package (/folder) contains the following files:
\begin{itemize}
    \item \code{\_\_init\_\_.py}: needed to make \code{lorenz} a package
    \item \code{run.py}: This script can be run to simulate the Lorenz attractor.  
    \item \code{solver.py}: Solves the system and returns vectors of data with $x$, $y$ and $z$ over time.
    \item \code{finiteDifferences.py}: Used by \code{solver.py} to solve the equations using Eqs. \eqref{eq:updateEqs}.  
    \item \code{plotting.py}: Plots the results. 
\end{itemize}
\vspace{1em}
The \code{run.py} file runs the \code{solve} function in \code{solver.py} and plots the results (vectors of data with $x$, $y$ and $z$ over time) using the \code{plotLorenz} function in \code{plotting.py} (without generating files). The initial conditions, parameters ($\sigma$, $\beta$, and $\rho$), time step $\td$ and number of iterations $N$ can be changed in the file. I decided to have a separate file \code{finiteDifferences.py} that calculates the finite difference schemes, and is used by the solver to have a better overview of what is happening.

The \code{plotting.py} file plots the values of $x$, $y$ and $z$ over time returned by the \code{solve} function in \code{solver.py}. Three of the plots are 2D from an $(x,y)$, $(x,z)$ and $(y,z)$ perspective, and one is a 3D plot. Apart from the data output from \code{solver.solve}, \code{plotting.plotLorenz} takes in some additional arguments. The first is the \code{step\_size} that allows to skip some of the indices of the $x$, $y$ and $z$ vectors as it might get `too crowded' in the plot for a smaller value of $\td$. Furthermore, the \code{scatter\_size} argument determines the size of the scattered dots. The first data-entry is plotted in red and denotes the initial condition. The rest of the data is colour-coded based on the current velocity of the system state (again, calculated using Euler's method) and a colourbar is added as a legend. 

The \code{solver.solve} and \code{plotting.plotLorenz} functions contain timers that print how long the functions took to run. This is to get an idea about the parts of the code that are most time-consuming.
\vspace{1em}

The \code{cases} folder contains the following files:
\begin{itemize}
    \item \code{run\_case.py}: Script that uses \code{generate\_data\_and\_figures.py} and generates the data and plots for a case specified in the code.
    \item \code{generate\_data\_and\_figures.py}: Called from \code{run\_case.py} and generates the data and plots for the case specified. It uses the \code{scheme\_variables.csv} file to retrieve the cases data.
    \item \code{generate\_cases\_CSV.py}: Used to generate the \code{scheme\_variables.csv} file.
    \item \code{scheme\_variables.csv}: Contining the cases data from Section \ref{sec:cases} generated by the  \code{generate\allowbreak\_cases\_CSV.py} file/.
    \item \code{output\_files}: Contains a folder for each case storing the plots and the data generated by \code{run\_case.py}
\end{itemize}
\vspace{1em}
I decided to not have separate files to generate each of the cases (i.e., \code{case1.py}, \code{case2.py}, etc.) but instead have one script \code{run\_case.py} where the case number can be changed in the script. This to avoid copy-pasting the same file 5 times. The \code{run\_case.py} file acts the same way as \code{run.py} but has the values of $\sigma$, $\beta$, and $\rho$ determined by the \code{scheme\_variables.csv} file rather than being user-determined. This \code{.csv} file contains the parameter-values of the 5 different cases presented in Section \ref{sec:cases}. An \code{action} variable can be changed from \code{`simulate'} to \code{`load'} to load the data after it has been generated. If there is no data available, an exception is thrown and the user is told to change \code{action} to \code{`simulate'} and generate the data first. Furthermore, the data and figures are generated and are saved in the \code{output\_files} folder in a(nother) folder called \code{case<casenumber>\_output}. So \code{case1\_output} for case 1, \code{case2\_output} for case 2, etc. The data generated (the output from \code{solver.solve}) is saved in an \code{.hdf5} file called \code{simulation\_result.hdf5}. I decided on this file-type, as it is a great way to store large amounts of data and work with in Python. Furthermore, the 4 plots generated by \code{plotting.py} are saved in \code{.pdf} format as it allows for vector images and is cross-platform. 

\section{Testing}
The main function that I want to test is \code{solver.solve} as this is where the ODE simulation is carried out. I chose to use \code{pytest} as the strategy for testing the code and decided against using \code{doctest}. This is because the function output is `complicated' -- large \code{numpy} arrays. I figured that a test class formatted for \code{pytest} can be written more elaborately than using the docstring and \code{doctest}.

Once I got plots that looked like the ones I found online, I knew that the implementation of the algorithm could be considered correct. In the future, to confirm this hunch, a comparison to the output of an ODE solver such as the one in \code{scipy} could be used to test whether the output of the approximation is close enough to the more accurate solution.

I decided against using a high number of iterations for testing and instead only used 2 iterations. This way I would know that the simulation had ran a few times and that it would go the right way and not return odd values. 

The first test I wanted to implement was one that checked whether if the the initial conditions are set to $0$ and the parameters are set to $0$, the solver should return $0$.

Then I tested with different initial conditions and parameter values, ran the solve function for 2 iterations and saved the output in the \code{test\_lorenz.py} file. This could then be used for testing the \code{solve} function at a later point and confirm that any changes to the algorithm would not affect the final result. 

As a wrong choice of initial conditions could make the scheme unstable, a final test should be implemented to check whether after a larger number of iterations (one that would normally be used to plot the results -- e.g. $N = 10000$), the values are \code{nan} (not-a-number). Unfortunately, I have not been able to implement this and this is left for future work. 

\bibliographystyle{IEEEtran}
\bibliography{references}

\end{document}
